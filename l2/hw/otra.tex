\documentclass[nohyper]{tufte-handout}

%\geometry{showframe}% for debugging purposes -- displays the margins
\usepackage[utf8]{inputenc}
\usepackage{amsmath}
\usepackage[spanish]{babel}
\usepackage[T1]{fontenc}
\usepackage{url}

\usepackage[unicode=true,pdfusetitle,
 bookmarks=true,bookmarksnumbered=true,bookmarksopen=true,bookmarksopenlevel=2,
 breaklinks=true,pdfborder={0 0 1},backref=false,colorlinks=false]
 {hyperref}
\hypersetup{
 pdfstartview=FitH}
%\usepackage{breakurl}
\makeatletter

% Set up the images/graphics package
\usepackage{graphicx, framed, xcolor}
\usepackage{colortbl}
%\setkeys{Gin}{width=\linewidth,totalheight=\textheight,keepaspectratio}
%\graphicspath{{graphics/}}
%\usepackage[table]{xcolor}
% The following package makes prettier tables.  We're all about the bling!
%\usepackage{booktabs}
\definecolor{MySalmon}{rgb}{1,0.85,0.85} 
\definecolor{shadecolor}{named}{MySalmon}
% The units package provides nice, non-stacked fractions and better spacing
% for units.
\usepackage{units}

%\usepackage{fancyvrb}
%\fvset{fontsize=\normalsize}

% Small sections of multiple columns
\usepackage{multicol}

% Provides paragraphs of dummy text
\usepackage{lipsum}
\decimalpoint


\makeatother

NA
NA

\usepackage{Sweave}
\begin{document}
\Sconcordance{concordance:otra.tex:otra.Rnw:%
1 48 1 1 0 9 1 1 6 57 1 1 2 1 0 3 1 11 0 1 4 6 0 %
1 2 135 1 1 8 12 0 1 2 111 1 1 2 1 0 1 1 3 0 1 2 %
128 1 1 2 15 0 1 2 1 1 1 2 1 0 2 1 7 0 1 2 2 1 1 %
2 7 0 1 1 1 2 1 0 2 1 3 0 1 2 126 1 1 5 7 0 1 2 %
4 1 1 2 10 0 1 2 1 1 1 2 10 0 1 2 94 1 1 2 1 0 2 %
1 7 0 1 2 1 1 1 2 1 0 1 1 7 0 1 1 8 0 1 2 6 1 1 %
2 1 0 3 1 7 0 1 1 8 0 1 2 3 1 1 2 9 0 1 2 41 1 1 %
2 1 0 3 1 7 0 1 1 8 0 1 2 8 1 1 2 1 0 3 1 7 0 1 %
1 8 0 1 2 6 1 1 13 60 1 1 2 1 0 4 1 11 0 1 1 3 0 %
1 2 8 1 1 2 1 0 2 1 3 0 1 2 4 1 1 2 1 0 6 1 3 0 %
1 2 90 1 1 2 1 0 1 1 5 0 1 1 5 0 1 1 1 6 8 0 1 2 %
3 1 1 2 1 0 5 1 1 2 4 0 1 2 12 1 1 2 1 0 1 4 6 0 %
1 2 2 1 1 2 1 0 2 1 6 0 1 2 3 1 1 2 6 0 1 1 28 0 %
1 2 3 1 1 2 1 0 1 1 5 0 1 1 15 0 1 1 28 0 1 2 10 %
1}


%\SweaveOpts{concordance=TRUE}
%\maketitle
%\normalsize
%\raggedright
%\setlength\parindent{10pt}
\maketitle



NA


NA
NA
NA

\begin{shaded}
NA
de otras variables de entrada $X= (X_1, X_2, \ldots, X_p)$. 
\end{shaded}

\subsection{Ejemplos}
NA
NA
NA

NA
NA
no es 'mayor' que 1, por ejemplo), sino que son {\em etiquetas} que distinguen patrones o resultados.
NA
escrito a mano.

NA

NA
NA
NA
dada por $f(X)$ con un estimador $\hat{f}(X)$. Cuanto mejor estimamos $f(X)$, mejor 
NA

NA
NA

NA
ciertas probabilidades de observar los distintos valores de $G$.

NA
$P(G=paga | X=95\%)=0.85$, mientras que $P(G=paga| X=15\% )=0.97$. En este caso, el 'ruido' o aleatoriedad se expresa en el hecho de que no sabemos con certidumbre el valor de $G$. La parte
NA

\begin{shaded}
{\bf Probabilidades condicionales de clase. }
NA
NA
las funciones
$$p_g(x) = P(G=g|X=x)$$ 
para cada clase $g$ que $G$ puede tomar, y cada posible valores de las entradas $x$.
\end{shaded}

\subsection{Ejemplo}
NA
NA
NA
NA
no se traslapen). 
\begin{Schunk}
\begin{Sinput}
> library(MASS)
> library(ggplot2)
> data(Pima.tr)
> head(Pima.tr)
\end{Sinput}
\begin{Soutput}
  npreg glu bp skin  bmi   ped age type
1     5  86 68   28 30.2 0.364  24   No
2     7 195 70   33 25.1 0.163  55  Yes
3     5  77 82   41 35.8 0.156  35   No
4     0 165 76   43 47.9 0.259  26   No
5     0 107 60   25 26.4 0.133  23   No
6     5  97 76   27 35.6 0.378  52  Yes
\end{Soutput}
\begin{Sinput}
> ggplot(Pima.tr, aes(x=glu, y=as.numeric(type)-1, colour=type, group=1)) + 
+   geom_jitter(position=position_jitter(height=0.05)) + 
+   geom_smooth(se = FALSE, span=0.8) + 
+   ylab('Probabilidad de tener diabetes')
\end{Sinput}
\end{Schunk}
NA
que da la probabilidad de que una persona tenga diabetes si su prueba de glucosa
NA
$$P(G=1 | X=x)=P(diabetes| Glucosa = x)$$


NA

NA
NA
queremos predecir. Sea entonces ${\mathcal L}=\{(x_1,g_1), (x_2,g_2),\ldots,(x_n,g_n)\}$ una muestra de entrenamiento.
Estimamos las funciones (probabilidades condicionales de clase) $p_g(x)$
mediante

$$\hat{p}_g(x_0)=\frac{1}{k}\sum_{(x_i,y_i)\in N_k(x_0)} I(g_i=g)$$
que es simplemente
$$\hat{p}_g (x_0)=\mbox{porcentaje de casos tipo g entre k vecinos de }x_0$$

NA
clase para estimar las probabilidades condicionales de clase.


\subsection{Ejercicio}
NA
NA
NA



\subsection{Clasificador de Bayes}

NA
NA
NA

NA
correcto. Es decir,

\begin{shaded}
NA
$${G}(x) = argmax_{g} P(G=g | X=x).$$
A este clasificador $G(x)$ se le llama {\bf clasificador de Bayes}.
\end{shaded}

\subsection{Ejemplo}
NA
usamos ciertas entradas $X=x$,  calculamos para una imagen particular que
$$P(G=gato|X=x)=0.45,$$ 
$$P(G=perro|X=x)=0.25,$$
$$P(G=persona|X=x)=0.02$$ 
NA
NA
clasificamos a la imagen como una que contiene un gato.

NA

NA
de clase es que esta forma de clasificar 
NA
incorrecta}

\begin{shaded}
NA
NA
$${G}(x) = argmax_{g} P(G=g | X=x).$$
NA
{\em tasa de Bayes}.
\end{shaded}


NA
NA


\begin{shaded}
NA
$$Err = E[I(\hat{G}\neq G)    ] = P(\hat{G}\neq G),$$
NA
\end{shaded}


Este error lo podemos estimar con una muestra de prueba. Si la muestra
de prueba es 
$${\mathcal T}=\{(x_1^0, g_1^0),(x_1^0, g_1^0),\ldots, (x_1^0, g_1^0) \},$$ entonces
$$\hat{Err} = \frac{1}{m}\sum_{i=1}^m I(g_i^0 \neq \hat{G}(x_i^0)).$$

NA
NA

NA
NA
NA
NA


NA
NA
en algunos problemas no todos los errores
son iguales ni deben costar lo mismo. 
NA
NA
cuando hay fraude. Cada error tiene distintos costos y consecuencias. En este
NA


NA
 
NA
NA

\begin{shaded}
\begin{itemize}
NA
$$\hat{G}(x) = argmax_{g} P(G=g | X=x),$$
para calcularlo se necesitan conocer las probabilidades a posterior de clase
$P(G=g|X=x)$ para cada clase $g$ y entrada $X=x$.
\item Usando una muestra de entrenamiento ${\mathcal L}=\{(x_i, g_i)  \}_{i=1}^n$ (muestra
etiquetada), construimos estimaciones
$$\hat{p}_g (x)$$
de cada $p_g(x)=P(G=g|X=x).$
\item Replicamos el primer inciso con las estimaciones, y construimos
el clasificador
$$\hat{G} (x) = argmax_g \hat{p}_g(x).$$
NA
NA
es el clasificador de Bayes.
\item Evaluamos el clasificador con una muestra de prueba, por ejemplo
NA
\end{itemize}
\end{shaded}


\subsection{Ejemplo}
Consideramos otra vez el ejemplo de diabetes. Supongamos
que obtenemos los siguientes resultados para la muestra de prueba
(prob.Yes y prob.No son las estimaciones de nuestro modelo):
\begin{Schunk}
\begin{Sinput}
> pima.ej
\end{Sinput}
\begin{Soutput}
  glu type prob.Yes prob.No
1 148  Yes     0.62    0.38
2  85   No     0.23    0.77
3  89   No     0.32    0.68
4  78  Yes     0.20    0.80
5 197  Yes     0.75    0.25
\end{Soutput}
\end{Schunk}
NA
NA


NA

NA
tenemos dos probabilidades $p_a(x)$ y $p_b(x)$. Clasificamos a {\em a}
cuando $p_a(x)>0.5$, y clasificamos a $b$ si $p_a(x) < 0.5$


\subsection{Ejercicio}
NA
NA
NA
condicionales. 

NA
usando datos de prueba {\em Pima.te}. 





NA

NA
NA
NA
que son hiperplanos lineales del espacio de entradas. 
NA


\includegraphics[height=5cm]{clas_lineal.png}
\includegraphics[height=5cm]{clas_nolineal.png}

NA

NA
(impago-pago, tiene enfermedad-no tiene enfermedad, etc.), es posible definir la variable
NA
$$Y=\begin{cases}
1 & \mbox{si } G=b\\
0 & \mbox{si } G=a
\end{cases}
$$

NA
NA
estableciendo

\begin{equation}\label{clasiflineal}
p_1(x)=\beta_0 + \sum_{j=1}^p \beta_kx_j.
\end{equation}
Clasificamos $\hat{Y}=1$ cuando $p_1(x)>0.5$ y $\hat{Y}=0$ cuando
 $p_1(x)<0.5$.
 
 El problema con este enfoque es que el lado derecho de \ref{clasiflineal} no
NA
 Esto no es un problema grave pero es indeseable. 
 
 Sin embargo, cuando tenemos
NA
NA
 
$$Y=\begin{cases}
0 & \mbox{si } G=a\\
1 & \mbox{si } G=b\\
2 & \mbox{si } G=c
\end{cases}
$$

NA
NA
NA
NA
lineales, una para cada clase (0-1), pero, igual que en el ejemplo de dos clases,
NA
probabilidades.

NA

NA
NA

NA
tenemos una sola entrada $X_1$. Supongamos entonces
que $G\in\{1,2\}$, y $Y=1$ si $G=2$, $Y=0$ si $G=1$. 

Vimos que estimar poniendo 
$$p_1(x)=\beta_0+\beta_1 x_1$$
tiene el defecto de que el lado derecho puede producir valores fuera
NA
que transforme la recta real al intervalo $[0,1]:$
$$p_1(x) = h(\beta_0+\beta_1 x_1),$$
NA
NA

NA
$\beta_0=0$ y $\beta_1=1$, de modo que
$$p_1(x)=h(x).$$
NA
No van a funcionar polinomios, por ejemplo, porque para un polinomio cuando
$x$ tiende a infinito, el polinomio tiende a $\infty$ o a $-\infty$.
NA
al margen):

\begin{shaded}
NA
$$h(x)=\frac{e^x}{1+e^x}$$
\end{shaded}
\begin{marginfigure}
\begin{Schunk}
\begin{Sinput}
> h <- function(x){exp(x)/(1+exp(x)) }
> curve(h, from=-6, to =6)
\end{Sinput}
\end{Schunk}
\end{marginfigure}

NA
el rango de todos los reales dentro del intervalo $[0,1]$.


Para $\beta_0$ y $\beta_1$ generales, 

\begin{shaded}
NA
$$p_1(x)=p_1(x;\beta)= h(\beta_0+\beta_1x_1)= \frac{e^{\beta_0+\beta_1x_1}}{1+ e^{\beta_0+\beta_1x_1}},$$
y $$p_0(x)=p_0(x;\beta)=1-p_1(x;\beta),$$
donde $\beta=(\beta_0,\beta_1)$.
\end{shaded}

NA

\subsection{Ejercicio}
\begin{itemize}
NA
NA
$$p_2(x)=\frac{1}{1+e^{\beta_0+\beta_1x_1}}.$$
\item Graficar las funciones $p_1(x;\beta)$ para distintos
valores de $\beta_0$ y $\beta_1$.
\end{itemize}

NA

Ahora consideramos el problema de ajustar el modelo anterior
cuando tenemos datos de entrenamiento 
$${\mathcal L}=\{ (x_1,y_1),(x_2,y_2),\ldots, (x_n,y_n) \},$$
NA
$\beta=(\beta_0,\beta_1)$ para reflejar lo mejor posible 
los datos de entrenamiento? Nuestro primer trabajo es
NA
datos de entrenamiento}.


NA

Un primer intento puede ser considerar como medida de desajuste del modelo
NA
$$L(\beta) = \frac{1}{n}\sum_{i=1}^n (y_i- p_1(x))^2$$
NA
NA

Si pensamos, por ejemplo, en el problema de tres clases,
NA
NA
a 1. De esta forma, vemos que una medida de desajuste de un modelo
NA
$$\frac{1}{n}\sum_{i=1}^n s(p_{g_i}(x))$$
NA
NA
$s$?

Si consideramos que observar la clase $g_i$ cuando $p_{g_i}(x)$ es muy cercana
NA
NA
$p_{g_i}(x)$ es cercana a cero. Entonces podemos tomar, por ejemplo
$s(z)=-\log(z)$.
NA
NA
como medida de desajuste
$$- \frac{1}{n}\sum_{i=1}^n \log (p_{g_i(x)}).$$

NA
NA

Supongamos que tenemos las funciones $p_g(x)$, y nuestra muestra
de entrenamiento $${\mathcal L}=\{ (x_1,g_1),(x_2,g_2),\ldots, (x_n,g_n) \},$$
(observaciones independientes)
Bajo el modelo, la probabilidad de observar la muestra ${\mathcal L}$ es
$$\prod_{i=1}^n p_{g_i}(x_i).$$
NA
NA
NA
maximizar la log-verosimilitud
$$\sum_{i=1}^n \log(p_{g_i}(x_i)),$$
que es equivalente a minimizar
$$- \frac{1}{n}\sum_{i=1}^n \log(p_{g_i}(x_i)),$$
NA
cuando observamos $g_i$ y nuestro modelo da la probabilidad $p_{g_i}(x)$ a la cantidad
NA



NA

Ahora regresamos a nuestro problema de dos clases ($y_i=1$ o $y_i=0$), con
NA

Recordamos entonces que

$$p_1(x)=p_1(x;\beta)= h(\beta_0+\beta_1x_1)= \frac{e^{\beta_0+\beta_1x_1}}{1+ e^{\beta_0+\beta_1x_1}},$$
y $$p_0(x)=p_0(x;\beta)=1-p_1(x;\beta),$$

Si nuestra muestra de entrenamiento es 
$${\mathcal L}=\{ (x_1,y_1),(x_2,y_2),\ldots, (x_n,y_n) \},$$

nuestro criterio de desajuste, la {\bf devianza} es 
NA
pues suponemos el conjunto de entrenamiento fijo):

$$D(\beta) = -\frac{2}{n}\sum_{i=1}^n log(p_{y_i}(x_i;\beta))$$

que podemos reescribir separando los casos $y_i=1$ y $y_i=0$:

\begin{shaded}
NA
la {\em devianza}, dada por
$$D(\beta) = -\frac{2}{n}\left(\sum_{y_i=1} \log(p_1(x_i;\beta) + \sum_{y_i=0}\log(1-p_1(x_i;\beta)))\right ).$$
\end{shaded}

 Se puede demostrar que este
NA
NA

NA

$$D(\beta) = -\frac{2}{n}\left(\sum_{i=1}^n
y_i\log(p_1(x_i;\beta)) + (1-y_i)\log(1-p_1(x_i;\beta))\right ).$$


\subsection{Ejemplo}
NA

NA

\begin{Schunk}
\begin{Sinput}
> glm(type ~ glu, data=Pima.tr, family = 'binomial')
\end{Sinput}
\begin{Soutput}
Call:  glm(formula = type ~ glu, family = "binomial", data = Pima.tr)

Coefficients:
(Intercept)          glu  
   -5.50364      0.03778  

Degrees of Freedom: 199 Total (i.e. Null);  198 Residual
Null Deviance:	    256.4 
Residual Deviance: 207.4 	AIC: 211.4
\end{Soutput}
\end{Schunk}

Y para los coeficientes estandarizados:
\begin{Schunk}
\begin{Sinput}
> Pima.tr$glu.st <- (Pima.tr$glu - mean(Pima.tr$glu))/sd(Pima.tr$glu)
> mod.1 <- glm(type ~ glu.st, data=Pima.tr, family = 'binomial')
> coef(mod.1)
\end{Sinput}
\begin{Soutput}
(Intercept)      glu.st 
 -0.8195882   1.1965055 
\end{Soutput}
\end{Schunk}

NA

\begin{Schunk}
\begin{Sinput}
> quantile(Pima.tr$glu.st)
\end{Sinput}
\begin{Soutput}
        0%        25%        50%        75%       100% 
-2.1463832 -0.7569340 -0.1095770  0.6325152  2.3693266 
\end{Soutput}
\begin{Sinput}
> grid.glu <- seq(-2.2, 2.4, 0.1)
> preds.grid <- predict(mod.1, newdata = data.frame(glu.st=grid.glu), 
+   type='response')
> dat.graf <- data.frame(glu.st=grid.glu, prob=preds.grid)
> ggplot(dat.graf, aes(x=glu.st, y = prob)) + geom_line()
\end{Sinput}
\end{Schunk}

NA



NA


Resolveremos este problema con descenso en gradiente. En primer lugar,
calculamos las derivadas de $D$ con respecto a $\beta_0$ y $\beta_1$,
NA
$$\frac{\partial  p_1}{\partial \beta_0} = {p_1(x;\beta)(1-p_1(x;\beta))},$$
y 
$$\frac{\partial  p_1}{\partial \beta_1} = {p_1(x;\beta)(1-p_1(x;\beta))x},$$
NA

$$ \frac{\partial D}{\partial\beta_0} = \frac{2}{n}\sum_{i=1}^n   (p(x_i;\beta)-y_i)$$
y 
$$ \frac{\partial D}{\partial\beta_1} = \frac{2}{n}\sum_{i=1}^n   (p(x_i;\beta)-y_i)x_i$$
NA

\begin{shaded}
$$\beta_0 = \beta_0 - \frac{2\eta}{n} \sum_{i=1}^n (p(x_i;\beta) -y_i)$$
$$\beta_1 = \beta_1 - \frac{2\eta}{n} \sum_{i=1}^n (p(x_i;\beta)-y_i)x_i$$
\end{shaded}
NA


NA

Consideremos ahora que tenemos entradas $X_1,X_2,\ldots, X_p$, y que la 
respuesta es $Y$ con posibles valores 0 o 1. Suponemos entonces que


$$p_1(x)=p_1(x;\beta)= h(\beta_0+\beta_1x_1+\cdots+\beta_px_p)= 
\frac{e^{\beta_0+\beta_1x_1+\beta_2x_2+\cdots+\beta_px_p}}{1+ e^{\beta_0+\beta_1x_1+\beta_2x_2+\cdots+\beta_px_p}},$$

Para calcular ahora las derivadas de $D$ con respecto a $\beta_0, \beta_1,\beta_2,\ldots, \beta_p$,
NA
$$\frac{\partial  p_1}{\partial \beta_0} = {p_1(x;\beta)(1-p_1(x;\beta))},$$
y 
$$\frac{\partial  p_1}{\partial \beta_j} = {p_1(x;\beta)(1-p_1(x;\beta))x_j},$$
NA

$$ \frac{\partial D}{\partial\beta_0} = \frac{2}{n}\sum_{i=1}^n   (p(x_i;\beta)-y_i)$$
y 
$$ \frac{\partial D}{\partial\beta_1} = \frac{2}{n}\sum_{i=1}^n   (p(x_i;\beta)-y_i)x_{ij}$$
NA

\begin{shaded}
$$\beta_0 = \beta_0 - \frac{2\eta}{n} \sum_{i=1}^n (p(x_i;\beta)-y_i)$$
$$\beta_j = \beta_j - \frac{2\eta}{n} \sum_{i=1}^n (p(x_i;\beta)-y_i)x_{ij}$$
para $j=1,\ldots, p$.
\end{shaded}

\begin{shaded}

Si definimos $x_{i0} = 1$ (equivale a agregar una columna de unos antes de las
NA
$$\beta_j = \beta_j - \frac{2\eta}{n} \sum_{i=1}^n (p(x_i;\beta)(x_i)-y_i)x_{ij}$$
para $j=0,1,\ldots, p$.
\end{shaded}

\subsection{Ejemplo}
NA
NA
NA
NA
NA

NA

Una vez que hemos estimado $\hat{p}_1 (x)$ y $\hat{p}_0 (x)$
NA
NA

NA
NA
$\hat{G}(x)=1$ si $\hat{p}_1 (x)>0.5$ y $hat{G}(x)=0$ en otro caso, o
$$\hat{G}(x)=argmax_{g=0,1} \hat{p}_g(x)$$
\end{shaded}

Con este clasificador $\hat{G}$ podemos evaluar la tasa
NA
o una muestra de prueba.

Falta por verificar que en efecto este proceso da un clasificador
NA
$$h(\hat{\beta}_0+\hat{\beta}_1x_1+\cdots+\hat{\beta}_px_p)>0.5.$$
Tomando la inversa de $h$, y notando que $h(0)=0.5$,
$$\hat{\beta}_0+\hat{\beta}_1x_1+\cdots+\hat{\beta}_px_p > 0$$
NA
$$\hat{\beta}_0+\hat{\beta}_1x_1+\cdots+\hat{\beta}_px_p = 0,$$
NA
NA



NA

NA
NA
NA
distintos costos e implicaciones en problemas particulares, como en el
NA
NA

NA
estas importancias, pero este enfoque tiene dos dificultades:
\begin{itemize}
NA
NA
en el proceso de ajuste.
\end{itemize}

Una alternativa en estos casos es analizar directamente el tipo de errores
NA


\begin{shaded}
NA
NA
\end{shaded}

\subsection{Ejemplo}
NA

\begin{Schunk}
\begin{Soutput}
         A   B   C
A.pred  50   2   0
B.pred  20 105  30
C.pred  20  10  30
\end{Soutput}
\end{Schunk}

NA
NA
Podemos ver esta tabla de distintas formas, por ejemplo, usando porcentajes
NA
\begin{Schunk}
\begin{Sinput}
> round(prop.table(tabla.1, 2),2)
\end{Sinput}
\begin{Soutput}
          A    B    C
A.pred 0.56 0.02 0.00
B.pred 0.22 0.90 0.50
C.pred 0.22 0.09 0.50
\end{Soutput}
\end{Schunk}
NA
NA
\begin{Schunk}
\begin{Sinput}
> round(prop.table(tabla.1, 1),2)
\end{Sinput}
\begin{Soutput}
          A    B    C
A.pred 0.96 0.04 0.00
B.pred 0.13 0.68 0.19
C.pred 0.33 0.17 0.50
\end{Soutput}
\end{Schunk}


NA

Supongamos que la variable a predecir es binaria, y llamaremos
a una de las clases como {\em positivo} y a otra {\em negativo}. 
NA
o que una persona tiene una enfermedad.

\begin{shaded}
Hay dos tipos de errores en un clasificador binario (positivo - negativo):
\begin{itemize}
\item Falsos positivos (fp): clasificar como positivo a un caso negativo.
\item Falsos negativos (fn): clasificar como negativo a un caso positivo.
\end{itemize}
A los casos clasificados correctamente les llamamos positivos verdaderos (pv)
y negativos verdaderos (nv).
\end{shaded}

NA

\begin{center}
\begin{tabular}{r|cc}
 & positivo & negativo \\
 \hline
positivo.pred & pv & fp \\
negativo.pred & fn & nv \\
\end{tabular}
\end{center}

NA
que tiene la mayor parte de los casos en la diagonal de la matriz
NA

NA
NA

\begin{itemize}
\item {\bf Tasa de falsos positivos} (o {\em fallout})
$$\frac{fp}{fp+nv}=\frac{fp}{negativos}$$
\item {\bf Tasa de falsos negativos}  (o {\em miss})
$$\frac{fn}{pv+fn}=\frac{fn}{positivos}$$
NA
$$\frac{nv}{fp+nv}=\frac{nv}{negativos}$$
NA
$$\frac{pv}{pv+fn}=\frac{pv}{positivos}$$ 
\end{itemize}

NA

\begin{itemize}
NA
$$\frac{vp}{vp+fp}=\frac{vp}{pred.positivo}$$
\item {\bf valor predictivo negativo}
$$\frac{vn}{fn+vn}=\frac{vp}{pred.negativo}$$

\end{itemize}

Generalmente tomamos dos cantidades de estas que reflejen comportamiento
del clasificador con casos negativos y con casos positivos.

\begin{shaded}
NA
Sensibilidad-Especificidad.
\end{shaded}

NA

\begin{itemize}
NA
NA
\end{itemize}

\subsection{Ejercicio}
NA
Por ejemplo:
NA
de especificidad y sensibilidad.
NA


\begin{shaded}
Cada clasificador tiene un balance distinto especificidad-sensibliidad. 
Muchas veces no escogemos clasificadores por la tasa de incorrectos solamente,
sino que intentamos buscar un balance adecuado entre el comportamiento de
NA
\end{shaded}

\subsection{Ejercicio}
NA
NA
NA
NA


\begin{Schunk}
\begin{Sinput}
> modelo.1 <- glm(type ~ glu, data = Pima.tr, family = 'binomial')
> preds.1 <- fitted(modelo.1) > 0.5
> head(preds.1)
\end{Sinput}
\begin{Soutput}
    1     2     3     4     5     6 
FALSE  TRUE FALSE  TRUE FALSE FALSE 
\end{Soutput}
\end{Schunk}

Calculamos sobre la muestra de entrenamiento:
\begin{Schunk}
\begin{Sinput}
> tab.confusion.entrena <- table(preds.1, Pima.tr$type)
> tab.confusion.entrena
\end{Sinput}
\begin{Soutput}
preds.1  No Yes
  FALSE 118  36
  TRUE   14  32
\end{Soutput}
\begin{Sinput}
> prop.table(tab.confusion.entrena, 2)
\end{Sinput}
\begin{Soutput}
preds.1        No       Yes
  FALSE 0.8939394 0.5294118
  TRUE  0.1060606 0.4705882
\end{Soutput}
\end{Schunk}

La especificidad de entrenamiento es 0.89 , y la sensibilidad
de prueba es de 0.46.

Para la muestra de prueba:


\begin{Schunk}
\begin{Sinput}
> probs.prueba <- predict(modelo.1, newdata=Pima.te, type='response')
> preds.prueba <- probs.prueba > 0.5
> tab.confusion.prueba <- table(preds.prueba, Pima.te$type)
> tab.confusion.prueba
\end{Sinput}
\begin{Soutput}
preds.prueba  No Yes
       FALSE 206  58
       TRUE   17  51
\end{Soutput}
\begin{Sinput}
> prop.table(tab.confusion.prueba, 2)
\end{Sinput}
\begin{Soutput}
preds.prueba         No        Yes
       FALSE 0.92376682 0.53211009
       TRUE  0.07623318 0.46788991
\end{Soutput}
\end{Schunk}

La especificidad (de prueba) es 0.92 , y la sensibilidad
(de prueba) es de 0.46. 
NA
\begin{Schunk}
\begin{Sinput}
> prop.table(tab.confusion.prueba, 1)
\end{Sinput}
\begin{Soutput}
preds.prueba       No      Yes
       FALSE 0.780303 0.219697
       TRUE  0.250000 0.750000
\end{Soutput}
\end{Schunk}

NA
(17+58)/332=0.22.


\section{Punto de corte para un clasificador binario.}

NA
un clasificador binario no es apropiado para nuestros fines?

Recordemos que una vez que hemos estimado con $\hat{p}_1 (x)$, 
NA

\begin{enumerate}
\item Predecir {\em positivo} si $\hat{p}_1 (x)>0.5$,
\item Predecir {\em negativo} si $\hat{p}_1 (x)<0.5$
\end{enumerate}

Esto sugiere una regla alternativa:

\begin{shaded}
Para $0<d<1$, podemos utilizar nuestras estimaciones
$\hat{p}_1 (x)$ para construir un clasificador alternativo poniendo:
\begin{enumerate}
\item Predecir {\em positivo} si $\hat{p}_1 (x)>d$,
\item Predecir {\em negativo} si $\hat{p}_1 (x)<d$
\end{enumerate}
\end{shaded}

Distintos valores de $d$ dan distintos perfiles de sensibilidad-especificidad
NA

\begin{itemize}
NA
NA
un caso es positivo para clasificarlo como positivo. Eso quiere decir
NA
NA
pues captamos menos de los verdaderos positivos.

Por ejemplo, si en el caso de diabetes incrementamos el punto
de corte a 0.7:
\begin{Schunk}
\begin{Sinput}
> probs.prueba <- predict(modelo.1, newdata=Pima.te, type='response')
> preds.prueba <- probs.prueba > 0.7
> tab.confusion.prueba <- table(preds.prueba, Pima.te$type)
> tab.confusion.prueba
\end{Sinput}
\begin{Soutput}
preds.prueba  No Yes
       FALSE 220  77
       TRUE    3  32
\end{Soutput}
\begin{Sinput}
> prop.table(tab.confusion.prueba, 2)
\end{Sinput}
\begin{Soutput}
preds.prueba         No        Yes
       FALSE 0.98654709 0.70642202
       TRUE  0.01345291 0.29357798
\end{Soutput}
\end{Schunk}
NA

NA
que un caso es negativo para clasificarlo como negativo. Esto
aumenta la sensibilidad, pero la especificidad baja.
\end{itemize}

Por ejemplo, si en el caso de diabetes incrementamos el punto
de corte a 0.3:
\begin{Schunk}
\begin{Sinput}
> probs.prueba <- predict(modelo.1, newdata=Pima.te, type='response')
> preds.prueba <- probs.prueba > 0.3
> tab.confusion.prueba <- table(preds.prueba, Pima.te$type)
> tab.confusion.prueba
\end{Sinput}
\begin{Soutput}
preds.prueba  No Yes
       FALSE 170  37
       TRUE   53  72
\end{Soutput}
\begin{Sinput}
> prop.table(tab.confusion.prueba, 2)
\end{Sinput}
\begin{Soutput}
preds.prueba        No       Yes
       FALSE 0.7623318 0.3394495
       TRUE  0.2376682 0.6605505
\end{Soutput}
\end{Schunk}
NA
(capturamos 0.66 a de los positivos).

\subsection{Espacio ROC de clasificadores}
NA
NA



{\bf Observaciones:}
\begin{enumerate}
NA
que tiene tasa de falsos positivos igual a 0 y sensibilidad igual a 1.

NA

NA
Esto implica que cuando veamos un caso positivo, la probabilidad de 'atinarle' es de $p$ (sensibilidad), y cuando vemos un negativo,
NA
NA
de la diagonal? Estos son clasificadores particularmente malos,
pues existen clasificadores con mejor especificidad y/o sensibilidad que son clasificadores al azar! Sin embargo, se puede construir un mejor clasificador volteando las predicciones, lo que cambia sensibilidad por tasa de falsos positivos.
\end{enumerate}






NA
de la tasa de incorrectos, el de corte 0.5. Sin embargo,
NA
otros.

\subsection{Ejercicio}
Utilizaremos las variables {\em glu, bp, skin, bmi, ped,age} para predecir
la variable {\em type} (si tiene diabetes o no), como en el ejercicio anterior. 
Como datos de entrenamiento
utilizaremos {\em Pima.tr} y de prueba {\em Pima.te}
\begin{enumerate}


NA
{\em glm} o descenso en gradiente. 
Si usas descenso en gradiente, recuerda estandarizar las variables antes de comenzar.

NA
NA
NA
NA
reporta los valores de sensibilidad y especificidad. Utiliza puntos de 
NA
NA
NA

\end{enumerate}



\section{Perfil de un clasificador binario y curvas ROC}

En lugar de examinar cada punto de corte por separado, podemos
NA
NA

Consideramos los
NA

\begin{Schunk}
\begin{Sinput}
> library(plyr)
> library(tabplot)
> Pima.te.2 <- Pima.te
> Pima.te.2$probs.prueba <- predict(modelo.1, newdata=Pima.te, type='response')
> head(arrange(Pima.te.2, desc(probs.prueba)))
\end{Sinput}
\begin{Soutput}
  npreg glu bp skin  bmi   ped age type probs.prueba
1     2 197 70   45 30.5 0.158  53  Yes    0.8743254
2     4 197 70   39 36.7 2.329  31   No    0.8743254
3     8 196 76   29 37.5 0.605  57  Yes    0.8701147
4     1 196 76   36 36.5 0.875  29  Yes    0.8701147
5     3 193 70   31 34.9 0.241  25  Yes    0.8567582
6     5 189 64   33 31.2 0.583  29  Yes    0.8371927
\end{Soutput}
\begin{Sinput}
> tableplot(Pima.te.2, sortCol=probs.prueba)
\end{Sinput}
\end{Schunk}
NA
podemos cortar para obtener diferentes clasificadores.
NA
NA

La calidad general del clasificador se puede resumir considerando,
NA

Por ejemplo, si usamos todas las variables, obtenemos:
\begin{Schunk}
\begin{Sinput}
> modelo.2 <- glm(type ~ ., data = Pima.tr[,2:8], family = 'binomial')
> Pima.te.2$probs.prueba.2 <- predict(modelo.2, newdata=Pima.te, type='response')
> tableplot(Pima.te.2, sortCol=probs.prueba.2)
\end{Sinput}
\end{Schunk}
NA

Podemos ahora mover el punto de corte de manera continua, y graficar todas las posibles combinaciones de especificidad y sensibilidad:


\begin{Schunk}
\begin{Sinput}
> library(ROCR)
> pred.rocr.1 <- prediction(Pima.te.2$probs.prueba, Pima.te.2$type)
> perf.1 <- performance(pred.rocr.1, measure='sens', x.measure='fpr')
> plot(perf.1)
> pred.rocr.2 <- prediction(Pima.te.2$probs.prueba.2, Pima.te.2$type)
> perf.2 <- performance(pred.rocr.2, measure='sens', x.measure='fpr')
> plot(perf.2, add=TRUE, col='red')
\end{Sinput}
\end{Schunk}

Este es un ejemplo de una curva ROC.

\begin{shaded}
NA
$\hat{p}_1(x)$, la curva ROC grafica todos los pares 
de (1-especificidad, sensibilidad) para cada posible punto de
corte $\hat{p}_1(x)>d$.
\end{shaded}


NA

NA
NA
que $G\in \{1,2,\ldots, K\}$ ($K$ clases), y tenemos
NA

Una estrategia es la de {\bf uno contra todos}:

NA
\begin{enumerate}
\item Para cada clase $g\in \{1,\ldots, K\}$ entrenamos un modelo
NA
\item Para clasificar un nuevo caso $x$, calculamos
$$\hat{p}_1^{(1)}(x), \hat{p}_1^{(2)}(x),\ldots, \hat{p}_1^{(K)}(x),$$
NA
$$\hat{G} (x) = argmax_g \hat{p}_1^{(g)}(x).$$
\end{enumerate}
\end{shaded}

NA
independientes de cada clase. En este sentido, produce estimaciones
que en realidad no satisfacen las propiedades del modelo de probabilidad establecido.
Sin embargo, {\em esta estrategia es simple y en muchos
casos funciona bien}.



NA

Si queremos obtener estimaciones de las probabilidades de clase que sumen
uno, entonces tenemos que contruir las estimaciones de cada clase
de clase de manera conjunta.

Como vimos antes, tenemos que estimar, para cada $x$ y $g\in \{1,\ldots,K\}$, las probabilidades condicionales de clase:
$$p_g(x)=P(G=g|X=x).$$ 

NA
$$\log P(G=1|X=x)= \beta_0+\beta_1x_1+\cdots+\beta_px_p - \log Z$$
$$\log P(G=2|X=x)=  - \log Z$$
donde vemos a $Z=1+\exp(\beta_0+\beta_1x_1+\cdots+\beta_px_p)$ como un 
NA
ponemos nuevos coeficientes para $G=2$ en estas ecuaciones, pues
$P(G=1|X=x)$ determina $P(G=2|X=x)=1-P(G=1|X=x)$.
 
NA
\begin{eqnarray*}
\log P(G=1|X=x)&= &\beta_0^1+\beta_1^1 x_1+\cdots+\beta_p^1x_p - \log  Z\\
\log P(G=2|X=x)&= &\beta_0^2+\beta_1^2 x_1+\cdots+\beta_p^2x_p - \log  Z\\
 &\vdots& \\
\log P(G=K|X=x) &= & - \log Z
\end{eqnarray*}

NA

Tomando exponencial, sumando, e igualando a uno (para garantizar que estas probabilidades suman 1), despejamos para
obtener $Z$:
$$Z = 1+ \sum_{j=1}^{K-1} \exp (\beta_0^j+\beta_1^jx_1+\cdots+\beta_p^jx_p), $$ 
de modo que
\begin{eqnarray*}
P(G=1|X=x)&= &
\frac{\exp(\beta_0^1+\beta_1^1 x_1+\cdots+\beta_p^1x_p)}{1+\sum_{j=1}^{K-1} \exp (\beta_0^j+\beta_1^jx_1+\cdots+\beta_p^jx_p)  }\\
P(G=2|X=x)&= &
\frac{\exp(\beta_0^2+\beta_1^2 x_1+\cdots+\beta_p^2x_p)}{1+\sum_{j=1}^{K-1} \exp (\beta_0^j+\beta_1^jx_1+\cdots+\beta_p^jx_p)  }\\
 &\vdots& \\
\log P(G=K|X=x) &= & \frac{1}{1+\sum_{j=1}^{K-1} \exp (\beta_0^j+\beta_1^jx_1+\cdots+\beta_p^jx_p)} 
\end{eqnarray*}

Para ajustar los coeficientes, utilizamos el criterio de devianza
que explicamos arriba. En este caso, buscamos minimizar la misma cantidad
$$D(\beta) = -\frac{2}{n}\sum_{i=1}^n p_{g_i}(x_i;\beta),$$
NA
$$\beta=(\beta_0^1, \beta_1^1, \ldots, \beta_p^1, \beta_0^2, \beta_1^2, \ldots, \beta_p^2,\ldots, \beta_0^{K-1}, \beta_1^{K-1}, \ldots, \beta_p^{K-1} ).$$

NA

NA

NA

\begin{Schunk}
\begin{Sinput}
> library(ElemStatLearn)
> dim(zip.train)
\end{Sinput}
\begin{Soutput}
[1] 7291  257
\end{Soutput}
\begin{Sinput}
> dim(zip.test)
\end{Sinput}
\begin{Soutput}
[1] 2007  257
\end{Soutput}
\begin{Sinput}
> mat.coeficientes <- matrix(NA, nrow=256+1, ncol=10)
> for(i in 0:9){
+   clase.bin <- as.numeric(zip.train[,1] == i)
+   dat <- data.frame(clase=clase.bin, zip.train[,-1])
+   modelo <- glm(clase ~ ., data=dat)
+   mat.coeficientes[,i + 1] <- coef(modelo)
+ }
\end{Sinput}
\end{Schunk}

NA
a matriz y graficando (eliminamos las $\beta_0$):
\begin{figure}
\begin{Schunk}
\begin{Sinput}
> library(reshape2)
> library(Hmisc)
> mat.2 <- melt(mat.coeficientes[-1,])
> mat.2$y <- -(mat.2$Var1 %/% 16 )
> mat.2$x <- mat.2$Var1 %% 16 
> mat.2$Var2 <- mat.2$Var2 - 1
> ggplot(mat.2, aes(x=x, y=y)) + 
+   geom_tile(aes(fill=cut2(value,g=3))) + facet_wrap(~Var2)
\end{Sinput}
\end{Schunk}
\end{figure}
NA
de que la tasa de error de prueba es casi el doble que la tasa
NA
 Por otro lado, un modelo simple como
NA
cada pixel contribuye de la misma forma independientemente de 
NA
NA
NA


Para construir predicciones utilizamos la matriz de coeficientes:
\begin{Schunk}
\begin{Sinput}
> h <- function(x){exp(x)/(1+exp(x))}
> pred.digito <- function(mat, beta){
+    prob.pred <- h(cbind(1, mat)%*%beta)
+    apply(prob.pred, 1, which.max) - 1
+ }
\end{Sinput}
\end{Schunk}

Por ejemplo, para la muestra de entrenamiento:

\begin{Schunk}
\begin{Sinput}
> library(ElemStatLearn)
> clasif.entrena <- pred.digito(zip.train[,-1], mat.coeficientes)
> head(clasif.entrena)
\end{Sinput}
\begin{Soutput}
[1] 6 5 4 7 3 4
\end{Soutput}
\end{Schunk}

NA
NA

\begin{Schunk}
\begin{Sinput}
> mean(clasif.entrena != zip.train[,1])
\end{Sinput}
\begin{Soutput}
[1] 0.07598409
\end{Soutput}
\begin{Sinput}
> table(clasif.entrena, zip.train[,1])
\end{Sinput}
\begin{Soutput}
clasif.entrena    0    1    2    3    4    5    6    7
             0 1177    0    9    8    2   17   10    4
             1    0 1002    5    0   25    1    7    4
             2    1    1  649    5    7    3    8    0
             3    4    0   22  612    0   30    1    0
             4    3    0   13    0  589   10   14    4
             5    1    0    2    9    3  478    9    2
             6    7    0   11    0    5    9  608    0
             7    0    0    6    9    1    0    0  593
             8    1    1   12   13    4    5    7    3
             9    0    1    2    2   16    3    0   35
              
clasif.entrena    8    9
             0   24    1
             1    6    7
             2    4    0
             3   17    1
             4   14   26
             5   12    6
             6    5    0
             7    3   20
             8  451    5
             9    6  578
\end{Soutput}
\end{Schunk}

NA
la muestra de prueba zip.test:

\begin{Schunk}
\begin{Sinput}
> clasif.prueba <- pred.digito(zip.test[,-1], mat.coeficientes)
> mean(clasif.prueba != zip.test[,1])
\end{Sinput}
\begin{Soutput}
[1] 0.1305431
\end{Soutput}
\begin{Sinput}
> table(clasif.prueba, zip.test[,1])
\end{Sinput}
\begin{Soutput}
clasif.prueba   0   1   2   3   4   5   6   7   8   9
            0 347   0   8   5   3   9   3   3   8   0
            1   2 254   3   0   7   1   2   1   3   3
            2   0   0 157   4   4   0   4   1   2   0
            3   1   2   7 138   0  20   0   1  14   0
            4   3   3   9   2 169   2   5   7   4   7
            5   1   0   2   8   1 122   4   0  10   0
            6   4   3   1   0   3   0 151   0   1   0
            7   0   0   1   2   2   1   0 130   1   7
            8   0   1  10   3   1   1   1   0 119   2
            9   1   1   0   4  10   4   0   4   4 158
\end{Soutput}
\begin{Sinput}
> round(prop.table(table(clasif.prueba, zip.test[,1]), margin = 2),2)
\end{Sinput}
\begin{Soutput}
clasif.prueba    0    1    2    3    4    5    6    7
            0 0.97 0.00 0.04 0.03 0.02 0.06 0.02 0.02
            1 0.01 0.96 0.02 0.00 0.04 0.01 0.01 0.01
            2 0.00 0.00 0.79 0.02 0.02 0.00 0.02 0.01
            3 0.00 0.01 0.04 0.83 0.00 0.12 0.00 0.01
            4 0.01 0.01 0.05 0.01 0.84 0.01 0.03 0.05
            5 0.00 0.00 0.01 0.05 0.00 0.76 0.02 0.00
            6 0.01 0.01 0.01 0.00 0.02 0.00 0.89 0.00
            7 0.00 0.00 0.01 0.01 0.01 0.01 0.00 0.88
            8 0.00 0.00 0.05 0.02 0.00 0.01 0.01 0.00
            9 0.00 0.00 0.00 0.02 0.05 0.02 0.00 0.03
             
clasif.prueba    8    9
            0 0.05 0.00
            1 0.02 0.02
            2 0.01 0.00
            3 0.08 0.00
            4 0.02 0.04
            5 0.06 0.00
            6 0.01 0.00
            7 0.01 0.04
            8 0.72 0.01
            9 0.02 0.89
\end{Soutput}
\end{Schunk}

NA
NA
NA


\section{Tarea}
Seguir y replicar el ejercicio de {\em Introduction to Statistical Learning with R} (ver archivo pdf en el sitio).  Haz curvas ROC para los clasificadores
NA
NA
NA
